cd ~/%!TEX root = ../dokumentation.tex


\chapter{Implementation}\label{cha:Implementation}

\section{Lexer}\label{sec:ImplementationLexer}

-Definition of tokens 

-Tabs and newlines ignored

-Newline would be helpful to identify comments because a comment is a newline followed by the percentage sign, as well as new rules if each rule would be represented in one line
However, there are rules that cover multiple lines. That is the main reason newlines are ignored. 

-A comment is identified by the lexer as a percentage sign followed by an arbitrary character excluding \dq ]\dq. This is followed by any arbitrary character. A comment can not only be identified by a percentage sign as the percentage sign is also part of the terminal symbols. However, the percentage symbol when used as terminal symbol is embedded in square brackets. 

Tokens:
LGRAMMAR/TOKEN/STRICT/MACRO EXPRESSION:

Any arbitrary symbol that is the name of the rule followed by the symbol itself (:==,:::,...)

Non terminal symbol:

A non terminal symbol starts with \dq <\dq and ends with \dq >\dq. In between there is any arbitrary sequence of numbers, underscores and small or captial letters.

T SYMBOL:

COMMENT:
        

OPEN SQUARE BRACKET/CLOSE SQUARE BRACKET, OPEN/CLOSE PARENTHESIS, ALTERNATIVE SYMBOL, REPETITION SYMBOL:

-is recognized and represented by the symbol itself

test
\section{Parser}\label{sec:ImplementationParser}

The parser is taking the tokens from the lexer and matches them to defined production rules.

\section{GUI}\label{sec:GUI}
