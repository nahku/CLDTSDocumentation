%!TEX root = ../dokumentation.tex

\chapter{Introduction}\label{cha:Introduction}

\section{Problem Statement and Goals}\label{sec:Aufgabenstellung}
Computer languages are likely to grow over time as they are getting more complex when their functionality is extended and more application cases are covered.
On the one hand that leads to a more powerful language.
However, on the other hand it becomes harder to understand the language and to implement it.
Thus, it becomes harder for new users to use the language.\\
This problem can be addressed by extracting smaller sub-languages for particular use cases. 
This could be done manually, but using this method is likely to raise errors or divergences from the original grammar.\\
Therefore, the approach considered in this report is to develop an application that is able to automatically extract sub-languages from a language.
A sub-language should be specified by the user using the application.\\
This report focusses on the \acf{TPTP} language for automated theorem proving.
Sub-languages of interest are for example a grammar just for \ac{CNF} or \ac{FOF}
The grammar of the language is provided in an \acf{EBNF}.\\
The first step to extract a sub-language is to build a lexer takes the grammar of the TPTP language as input and generates tokens.
Subsequently a parser should build a parse tree that represents the grammar rules of the \ac{TPTP} language.
This parse tree should be visually presented to the user and the user can then choose which grammar rules should not be included in the desired sub-language.
After the user specified the sub-language, the developed application should extract the sub-language from the \ac{TPTP} language and present the sub-language in the same format as the original \ac{TPTP} syntax.
Also, comments present in the \ac{TPTP} syntax should be maintained and associated with the corresponing rules in the reduced syntax.
\section{Structure of the Report}\label{sec:IntroductionStructure}
This report is structured into six chapters. In the first chapter the research problem and goals of this research are stated.
Then in chapter 2 the necessary background information for the following chapters 3 and 4 is provided.
In chapter 3 the concept for the TODO APLICATION is developed.
Based on this, the implementation of the application is featured in chapter 4.
In chapter 5 the results of the TODO reduced grammar is tested on a problem which is presented in a form corresponding to the reduced grammar.
Chapter 6 sums up the results achieved in this research and offers an outlook for possible future research.