%!TEX root = ../dokumentation.tex

\chapter{Introduction}\label{cha:Introduction}

\section{Problem statement and goals}\label{sec:Aufgabenstellung}
Computer languages are likely to grow over time as they are getting more complex when their functionality is extended and more use cases are covered.
On one hand that leads to a more powerful language.
On the other hand increased complexity makes a language harder for new users to learn and to use.\\
The goal is to maintain the advantages of an expressive? language but allow users to extract a sub-syntax from a larger syntax to simplify the language for their particular use case. \\
Extracting a sub-syntax manually raises errors?may lead to? or divergences from the original grammar.\\
Therefore, this report describes a tool that is able to automatically extract a sub-syntax that has been specified by the user using the application.\\
This report focusses on the \acf{TPTP} language for automated theorem proving that has been expanded over time to include various classic logics.
Sub-syntaxes of interest are for example \ac{CNF} or \ac{FOF}.
The \ac{TPTP} syntax is provided in an \acf{EBNF}.\\
??The developed tool processes a given grammar in multiple steps.
First it parses the formal grammar into a structured internal representation using \ac{PLY}.
The processed grammar is presented to the user via a GUI. The user can select a start symbol and disable productions that should not be included in the desired sub-syntax.
Using the users input, the developed application extracts the sub-syntax from the \ac{TPTP} syntax and presents the sub-syntax in the same format as the original \ac{TPTP} syntax.
Also, comments present in the \ac{TPTP} syntax should be maintained and associated with the corresponing rules in the reduced syntax.
\section{Structure of the Report}\label{sec:IntroductionStructure}
The report is structured into six chapters. The first chapter introduces the problem of complex computer languages and the goal of this report that is extracting smaller sub-languages.
The second chapter provides necessary background information including the \ac{TPTP} language, formal grammars, lexing and parsing. By means of the background information, the third chapter outlines the concept of the developed tool.
Based on this, the implementation of the tool is featured in the fourth chapter.
The fifth chapter presents an evaluation of the effectiveness? of the tool. Considering the evaluation, the sixth chapter sums up the results of the developed tool, compares the results to the defined goals in first chapter and offers an outlook for possible future research.