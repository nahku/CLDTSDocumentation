%!TEX root = ../dokumentation.tex

\chapter{Background and Theory}\label{cha:Background}

\section{TPTP Language}\label{sec:BackgroundTPTP}
\cite{Sut17}

\section{\acf{BNF}}\label{sec:BackgroundBNF}

\section{Parser}\label{sec:BackgroundParser}

\subsection{Lex}\label{sec:Back groundLex}

Lexing/lexical analysis: Division of input into units so called tokens \cite{LexYacc.1992}

Input: description of tokens - lex specification, regular expressions]\cite{LexYacc.1992}
Output: routine that identifies those tokens \cite{LexYacc.1992}

\subsection{Yacc}\label{sec:BackgroundYacc}

Parsing: establish relationship among tokens \cite{LexYacc.1992}
Grammar: list of rules that defines the relationships \cite{LexYacc.1992}

Input: description of grammar \cite{LexYacc.1992}
Output: parser \cite{LexYacc.1992}


\subsection{PLY}\label{sec:BackgroundPLY}

Python implementation of lex and yacc
[LALR-parsing]
consists of lex.py and yacc.py

lex.py tokenizes an input string

http://www.dabeaz.com/ply/ply.html