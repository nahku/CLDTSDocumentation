%!TEX root = ../dokumentation.tex

\chapter{Conclusion}\label{cha:Conclusion}

The aim of this research was to create a tool that automatically extracts sub-syntaxes of the \ac{TPTP} syntax. ?to reduce entry barrier to new users and also maintain consistency with original grammar  ?..

The developed software tool enables users to consistently, conveniently, and automatically extract sub-languages from the \ac{TPTP} syntax, turning this one language into a family of languages, coming from a single source, but optimized for particular use cases.

This research included:
The analysis of the \ac{TPTP} syntax and the lexer and parser creation.
The conception and successful implementation of a sub-syntax extraction process.
The development of a format to describe what part of syntaxes to extract (control file format).
-Heuristic for comment association has been developed, to maintain comments from the original syntax in sub-syntaxes
-Sub-syntax generation
-Command-line interface, in which main operations can be performed without the necessity of using the GUI or ?installing? the package necessary for the GUI
-GUI which provides a graphical overview and aids the user in the process of sub-syntax extraction and control file generation.

The functionality of the tool was demonstrated by extracting \ac{CNF} sub-syntax and creating a parser based on that syntax, which has been successfully tested with ?a number? of \ac{TPTP} problems.

All requirements (section \ref{sec:ConceptRequirements}) have been fulfilled, and in addition useful functions like ?(importing a \ac{TPTP} syntax file from the \ac{TPTP} website)? also have been implemented. Therefore the goals of this research have been achieved.
\section{Future Work}\label{sec:FutureWork}
The concept of sub-syntax extraction is not limited to the \ac{TPTP} language. (Might be useful for any number of computer languages)
It is conceivable to extend the tool to apply it on other languages, e.g. SMT-lib~\cite{BFT-SMTLIB-17}.
For that, mainly the lexer and parser components would have to be adapted to parser the new syntax. The logic of the sub-syntax extraction would mostly remain the same.\\
Another option would be to create a tool, that accepts plain \ac{BNF} or \ac{EBNF} since many computer languages are described in these forms, which would ?provide? a standard tool that can be used with ?any number? of computer languages.

Based on their grammar, the lexing and parsing component have to be adapted to parse the given syntax.
The other components can be used with only minor modifications.
The graph generation for example needs to be adapted depending on the amount of production symbols.
Also comment association has to be adapted based on the specific comment convention.\\
Also it can be invesitgated, if comment association based on comment content would lead to better a significantly better result.
