%!TEX root = ../dokumentation.tex

%TODO: Einleitungen überarbeiten
\chapter{Concept}\label{cha:Concept}
This chapter outlines the concept and the architecture of the software tool. First, the proposed software architecture is described.



\section{Overview}\label{sec:ConceptOverview}

\subsection{Proposed Architecture}\label{sec:ConceptProposedArchitecture}

\section{Lexer}

todo In plain \ac{BNF} there are terminal, non terminal and production symbols

-deviations from plain ebnf
The \ac{TPTP} language is specified in an \ac{EBNF}, but is modified for its specific purpose???.
Therefore there are deviations from standard \ac{EBNF} that need to be analysed to spefify todo elementary tokens and rules for matching tokens.
The standard \ac{EBNF} uses only only one production symbol ("::="). In the \ac{TPTP} language
additional production symbols have been added. The following table \ref{tbl:ConceptTPTPProductionSymbols} contains the production symbols used in the \ac{TPTP} language.

\begin{table}[H]
\centering
\renewcommand{\arraystretch}{1}
\caption{\ac{TPTP} language production symbols \cite{VS06}}
\begin{tabular}{ll}
\textbf{Symbol} & \textbf{Rule Type}\\\hline
::= & Grammar\\
:== & Strict\\
::- & Token\\
::: & Macro\\
\end{tabular}
\label{tbl:ConceptTPTPProductionSymbols}
\end{table}



\section{Parser}
\subsection{Data Structure}
The \ac{TPTP} grammar, extracted from the \ac{TPTP} grammar file, needs to be stored in a data structure that allows for modification. A graph representing
possible transitions within the 
\section{Generation of the Reduced Grammar}\label{sec:ConceptGenerateReducedGrammar}

\section{Selection of blocked Productions}

\section{Determination of the remaining reachable Productions}

\section{Determination of  the remaining terminating Productions}


bei tree building temporäres startsymbol nutzen (da mehrere Startsymbole möglich)

\section{GUI}\label{sec:ConceptGUI}


