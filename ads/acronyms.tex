%!TEX root = ../dokumentation.tex

\addchap{\langabkverz}
%nur verwendete Akronyme werden letztlich im Abkürzungsverzeichnis des Dokuments angezeigt
%Verwendung: 
%		\ac{Abk.}   --> fügt die Abkürzung ein, beim ersten Aufruf wird zusätzlich automatisch die ausgeschriebene Version davor eingefügt bzw. in einer Fußnote (hierfür muss in header.tex \usepackage[printonlyused,footnote]{acronym} stehen) dargestellt
%		\acs{Abk.}   -->  fügt die Abkürzung ein
%		\acf{Abk.}   --> fügt die Abkürzung UND die Erklärung ein
%		\acl{Abk.}   --> fügt nur die Erklärung ein
%		\acp{Abk.}  --> gibt Plural aus (angefügtes 's'); das zusätzliche 'p' funktioniert auch bei obigen Befehlen
%	siehe auch: http://golatex.de/wiki/%5Cacronym
%	
\begin{acronym}[YTMMM]
\setlength{\itemsep}{-\parsep}
%\acrodefplural{AABB}[AABBs]{Axis-Aligned Bounding Boxes}
\acro{ATP}{Automated Theorem Proving}
\acro{BNF}{Backus-Naur Form}
\acro{CFG}{Context-free grammar}
\acrodefplural{CFG}[CFGs]{context-free grammars}
\acro{CNF}{Clause Normal Form}
\acro{EBNF}{Extended Backus-Naur Form}
\acro{FOF}{First-order Form}
\acro{PLY}{Python Lex-Yacc}
\acro{TFF}{Typed First-order Form}
\acro{THF}{Typed Higher-order Form}
\acro{TPTP}{Thousands of Problems for Theorem Provers}


\end{acronym}
