%!TEX root = ../dokumentation.tex

\pagestyle{empty}

\iflang{de}{%
% Dieser deutsche Teil wird nur angezeigt, wenn die Sprache auf Deutsch eingestellt ist.
\renewcommand{\abstractname}{\langabstract} % Text für Überschrift

% \begin{otherlanguage}{english} % auskommentieren, wenn Abstract auf Deutsch sein soll
\begin{abstract}

\end{abstract}
% \end{otherlanguage} % auskommentieren, wenn Abstract auf Deutsch sein soll
}



\iflang{en}{%
% Dieser englische Teil wird nur angezeigt, wenn die Sprache auf Englisch eingestellt ist.
\renewcommand{\abstractname}{\langabstract} % Text für Überschrift

\begin{abstract}
Computer languages are likely to grow over time as they are getting more complex when their functionality is extended. An example for that is the TPTP language for automated theorem proving. Over time various forms of classical logics ranging from Clause Normal Form (CNF) to Typed First-order Form (TFF) have been included in and extended the TPTP language.
This research describes a tool called Synplifier (Syntax simplifier) that automatically extracts sub-languages from the TPTP language.
Automatic extraction instead of manually maintaining sub-languages has the advantage of avoiding maintenance overhead as well as unnoticed divergences from the full language.
Sub-languages of interest are for example CNF or First-order Form (FOF) and are extracted based on the user's selection which part of the language to maintain.
Synplifier has been successfully tested by extracting CNF from the TPTP language.
\end{abstract}
}